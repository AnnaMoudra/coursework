
% Default to the notebook output style

    


% Inherit from the specified cell style.




    
\documentclass[11pt]{article}

    
    
    \usepackage[T1]{fontenc}
    % Nicer default font (+ math font) than Computer Modern for most use cases
    \usepackage{mathpazo}

    % Basic figure setup, for now with no caption control since it's done
    % automatically by Pandoc (which extracts ![](path) syntax from Markdown).
    \usepackage{graphicx}
    % We will generate all images so they have a width \maxwidth. This means
    % that they will get their normal width if they fit onto the page, but
    % are scaled down if they would overflow the margins.
    \makeatletter
    \def\maxwidth{\ifdim\Gin@nat@width>\linewidth\linewidth
    \else\Gin@nat@width\fi}
    \makeatother
    \let\Oldincludegraphics\includegraphics
    % Set max figure width to be 80% of text width, for now hardcoded.
    \renewcommand{\includegraphics}[1]{\Oldincludegraphics[width=.8\maxwidth]{#1}}
    % Ensure that by default, figures have no caption (until we provide a
    % proper Figure object with a Caption API and a way to capture that
    % in the conversion process - todo).
    \usepackage{caption}
    \DeclareCaptionLabelFormat{nolabel}{}
    \captionsetup{labelformat=nolabel}

    \usepackage{adjustbox} % Used to constrain images to a maximum size 
    \usepackage{xcolor} % Allow colors to be defined
    \usepackage{enumerate} % Needed for markdown enumerations to work
    \usepackage{geometry} % Used to adjust the document margins
    \usepackage{amsmath} % Equations
    \usepackage{amssymb} % Equations
    \usepackage{textcomp} % defines textquotesingle
    % Hack from http://tex.stackexchange.com/a/47451/13684:
    \AtBeginDocument{%
        \def\PYZsq{\textquotesingle}% Upright quotes in Pygmentized code
    }
    \usepackage{upquote} % Upright quotes for verbatim code
    \usepackage{eurosym} % defines \euro
    \usepackage[mathletters]{ucs} % Extended unicode (utf-8) support
    \usepackage[utf8x]{inputenc} % Allow utf-8 characters in the tex document
    \usepackage{fancyvrb} % verbatim replacement that allows latex
    \usepackage{grffile} % extends the file name processing of package graphics 
                         % to support a larger range 
    % The hyperref package gives us a pdf with properly built
    % internal navigation ('pdf bookmarks' for the table of contents,
    % internal cross-reference links, web links for URLs, etc.)
    \usepackage{hyperref}
    \usepackage{longtable} % longtable support required by pandoc >1.10
    \usepackage{booktabs}  % table support for pandoc > 1.12.2
    \usepackage[inline]{enumitem} % IRkernel/repr support (it uses the enumerate* environment)
    \usepackage[normalem]{ulem} % ulem is needed to support strikethroughs (\sout)
                                % normalem makes italics be italics, not underlines
    \usepackage{mathrsfs}
    

    
    
    % Colors for the hyperref package
    \definecolor{urlcolor}{rgb}{0,.145,.698}
    \definecolor{linkcolor}{rgb}{.71,0.21,0.01}
    \definecolor{citecolor}{rgb}{.12,.54,.11}

    % ANSI colors
    \definecolor{ansi-black}{HTML}{3E424D}
    \definecolor{ansi-black-intense}{HTML}{282C36}
    \definecolor{ansi-red}{HTML}{E75C58}
    \definecolor{ansi-red-intense}{HTML}{B22B31}
    \definecolor{ansi-green}{HTML}{00A250}
    \definecolor{ansi-green-intense}{HTML}{007427}
    \definecolor{ansi-yellow}{HTML}{DDB62B}
    \definecolor{ansi-yellow-intense}{HTML}{B27D12}
    \definecolor{ansi-blue}{HTML}{208FFB}
    \definecolor{ansi-blue-intense}{HTML}{0065CA}
    \definecolor{ansi-magenta}{HTML}{D160C4}
    \definecolor{ansi-magenta-intense}{HTML}{A03196}
    \definecolor{ansi-cyan}{HTML}{60C6C8}
    \definecolor{ansi-cyan-intense}{HTML}{258F8F}
    \definecolor{ansi-white}{HTML}{C5C1B4}
    \definecolor{ansi-white-intense}{HTML}{A1A6B2}
    \definecolor{ansi-default-inverse-fg}{HTML}{FFFFFF}
    \definecolor{ansi-default-inverse-bg}{HTML}{000000}

    % commands and environments needed by pandoc snippets
    % extracted from the output of `pandoc -s`
    \providecommand{\tightlist}{%
      \setlength{\itemsep}{0pt}\setlength{\parskip}{0pt}}
    \DefineVerbatimEnvironment{Highlighting}{Verbatim}{commandchars=\\\{\}}
    % Add ',fontsize=\small' for more characters per line
    \newenvironment{Shaded}{}{}
    \newcommand{\KeywordTok}[1]{\textcolor[rgb]{0.00,0.44,0.13}{\textbf{{#1}}}}
    \newcommand{\DataTypeTok}[1]{\textcolor[rgb]{0.56,0.13,0.00}{{#1}}}
    \newcommand{\DecValTok}[1]{\textcolor[rgb]{0.25,0.63,0.44}{{#1}}}
    \newcommand{\BaseNTok}[1]{\textcolor[rgb]{0.25,0.63,0.44}{{#1}}}
    \newcommand{\FloatTok}[1]{\textcolor[rgb]{0.25,0.63,0.44}{{#1}}}
    \newcommand{\CharTok}[1]{\textcolor[rgb]{0.25,0.44,0.63}{{#1}}}
    \newcommand{\StringTok}[1]{\textcolor[rgb]{0.25,0.44,0.63}{{#1}}}
    \newcommand{\CommentTok}[1]{\textcolor[rgb]{0.38,0.63,0.69}{\textit{{#1}}}}
    \newcommand{\OtherTok}[1]{\textcolor[rgb]{0.00,0.44,0.13}{{#1}}}
    \newcommand{\AlertTok}[1]{\textcolor[rgb]{1.00,0.00,0.00}{\textbf{{#1}}}}
    \newcommand{\FunctionTok}[1]{\textcolor[rgb]{0.02,0.16,0.49}{{#1}}}
    \newcommand{\RegionMarkerTok}[1]{{#1}}
    \newcommand{\ErrorTok}[1]{\textcolor[rgb]{1.00,0.00,0.00}{\textbf{{#1}}}}
    \newcommand{\NormalTok}[1]{{#1}}
    
    % Additional commands for more recent versions of Pandoc
    \newcommand{\ConstantTok}[1]{\textcolor[rgb]{0.53,0.00,0.00}{{#1}}}
    \newcommand{\SpecialCharTok}[1]{\textcolor[rgb]{0.25,0.44,0.63}{{#1}}}
    \newcommand{\VerbatimStringTok}[1]{\textcolor[rgb]{0.25,0.44,0.63}{{#1}}}
    \newcommand{\SpecialStringTok}[1]{\textcolor[rgb]{0.73,0.40,0.53}{{#1}}}
    \newcommand{\ImportTok}[1]{{#1}}
    \newcommand{\DocumentationTok}[1]{\textcolor[rgb]{0.73,0.13,0.13}{\textit{{#1}}}}
    \newcommand{\AnnotationTok}[1]{\textcolor[rgb]{0.38,0.63,0.69}{\textbf{\textit{{#1}}}}}
    \newcommand{\CommentVarTok}[1]{\textcolor[rgb]{0.38,0.63,0.69}{\textbf{\textit{{#1}}}}}
    \newcommand{\VariableTok}[1]{\textcolor[rgb]{0.10,0.09,0.49}{{#1}}}
    \newcommand{\ControlFlowTok}[1]{\textcolor[rgb]{0.00,0.44,0.13}{\textbf{{#1}}}}
    \newcommand{\OperatorTok}[1]{\textcolor[rgb]{0.40,0.40,0.40}{{#1}}}
    \newcommand{\BuiltInTok}[1]{{#1}}
    \newcommand{\ExtensionTok}[1]{{#1}}
    \newcommand{\PreprocessorTok}[1]{\textcolor[rgb]{0.74,0.48,0.00}{{#1}}}
    \newcommand{\AttributeTok}[1]{\textcolor[rgb]{0.49,0.56,0.16}{{#1}}}
    \newcommand{\InformationTok}[1]{\textcolor[rgb]{0.38,0.63,0.69}{\textbf{\textit{{#1}}}}}
    \newcommand{\WarningTok}[1]{\textcolor[rgb]{0.38,0.63,0.69}{\textbf{\textit{{#1}}}}}
    
    
    % Define a nice break command that doesn't care if a line doesn't already
    % exist.
    \def\br{\hspace*{\fill} \\* }
    % Math Jax compatibility definitions
    \def\gt{>}
    \def\lt{<}
    \let\Oldtex\TeX
    \let\Oldlatex\LaTeX
    \renewcommand{\TeX}{\textrm{\Oldtex}}
    \renewcommand{\LaTeX}{\textrm{\Oldlatex}}
    % Document parameters
    % Document title
    \title{MI-SPI Domácí úkol 2}
    
    
    
    
    

    % Pygments definitions
    
\makeatletter
\def\PY@reset{\let\PY@it=\relax \let\PY@bf=\relax%
    \let\PY@ul=\relax \let\PY@tc=\relax%
    \let\PY@bc=\relax \let\PY@ff=\relax}
\def\PY@tok#1{\csname PY@tok@#1\endcsname}
\def\PY@toks#1+{\ifx\relax#1\empty\else%
    \PY@tok{#1}\expandafter\PY@toks\fi}
\def\PY@do#1{\PY@bc{\PY@tc{\PY@ul{%
    \PY@it{\PY@bf{\PY@ff{#1}}}}}}}
\def\PY#1#2{\PY@reset\PY@toks#1+\relax+\PY@do{#2}}

\expandafter\def\csname PY@tok@w\endcsname{\def\PY@tc##1{\textcolor[rgb]{0.73,0.73,0.73}{##1}}}
\expandafter\def\csname PY@tok@c\endcsname{\let\PY@it=\textit\def\PY@tc##1{\textcolor[rgb]{0.25,0.50,0.50}{##1}}}
\expandafter\def\csname PY@tok@cp\endcsname{\def\PY@tc##1{\textcolor[rgb]{0.74,0.48,0.00}{##1}}}
\expandafter\def\csname PY@tok@k\endcsname{\let\PY@bf=\textbf\def\PY@tc##1{\textcolor[rgb]{0.00,0.50,0.00}{##1}}}
\expandafter\def\csname PY@tok@kp\endcsname{\def\PY@tc##1{\textcolor[rgb]{0.00,0.50,0.00}{##1}}}
\expandafter\def\csname PY@tok@kt\endcsname{\def\PY@tc##1{\textcolor[rgb]{0.69,0.00,0.25}{##1}}}
\expandafter\def\csname PY@tok@o\endcsname{\def\PY@tc##1{\textcolor[rgb]{0.40,0.40,0.40}{##1}}}
\expandafter\def\csname PY@tok@ow\endcsname{\let\PY@bf=\textbf\def\PY@tc##1{\textcolor[rgb]{0.67,0.13,1.00}{##1}}}
\expandafter\def\csname PY@tok@nb\endcsname{\def\PY@tc##1{\textcolor[rgb]{0.00,0.50,0.00}{##1}}}
\expandafter\def\csname PY@tok@nf\endcsname{\def\PY@tc##1{\textcolor[rgb]{0.00,0.00,1.00}{##1}}}
\expandafter\def\csname PY@tok@nc\endcsname{\let\PY@bf=\textbf\def\PY@tc##1{\textcolor[rgb]{0.00,0.00,1.00}{##1}}}
\expandafter\def\csname PY@tok@nn\endcsname{\let\PY@bf=\textbf\def\PY@tc##1{\textcolor[rgb]{0.00,0.00,1.00}{##1}}}
\expandafter\def\csname PY@tok@ne\endcsname{\let\PY@bf=\textbf\def\PY@tc##1{\textcolor[rgb]{0.82,0.25,0.23}{##1}}}
\expandafter\def\csname PY@tok@nv\endcsname{\def\PY@tc##1{\textcolor[rgb]{0.10,0.09,0.49}{##1}}}
\expandafter\def\csname PY@tok@no\endcsname{\def\PY@tc##1{\textcolor[rgb]{0.53,0.00,0.00}{##1}}}
\expandafter\def\csname PY@tok@nl\endcsname{\def\PY@tc##1{\textcolor[rgb]{0.63,0.63,0.00}{##1}}}
\expandafter\def\csname PY@tok@ni\endcsname{\let\PY@bf=\textbf\def\PY@tc##1{\textcolor[rgb]{0.60,0.60,0.60}{##1}}}
\expandafter\def\csname PY@tok@na\endcsname{\def\PY@tc##1{\textcolor[rgb]{0.49,0.56,0.16}{##1}}}
\expandafter\def\csname PY@tok@nt\endcsname{\let\PY@bf=\textbf\def\PY@tc##1{\textcolor[rgb]{0.00,0.50,0.00}{##1}}}
\expandafter\def\csname PY@tok@nd\endcsname{\def\PY@tc##1{\textcolor[rgb]{0.67,0.13,1.00}{##1}}}
\expandafter\def\csname PY@tok@s\endcsname{\def\PY@tc##1{\textcolor[rgb]{0.73,0.13,0.13}{##1}}}
\expandafter\def\csname PY@tok@sd\endcsname{\let\PY@it=\textit\def\PY@tc##1{\textcolor[rgb]{0.73,0.13,0.13}{##1}}}
\expandafter\def\csname PY@tok@si\endcsname{\let\PY@bf=\textbf\def\PY@tc##1{\textcolor[rgb]{0.73,0.40,0.53}{##1}}}
\expandafter\def\csname PY@tok@se\endcsname{\let\PY@bf=\textbf\def\PY@tc##1{\textcolor[rgb]{0.73,0.40,0.13}{##1}}}
\expandafter\def\csname PY@tok@sr\endcsname{\def\PY@tc##1{\textcolor[rgb]{0.73,0.40,0.53}{##1}}}
\expandafter\def\csname PY@tok@ss\endcsname{\def\PY@tc##1{\textcolor[rgb]{0.10,0.09,0.49}{##1}}}
\expandafter\def\csname PY@tok@sx\endcsname{\def\PY@tc##1{\textcolor[rgb]{0.00,0.50,0.00}{##1}}}
\expandafter\def\csname PY@tok@m\endcsname{\def\PY@tc##1{\textcolor[rgb]{0.40,0.40,0.40}{##1}}}
\expandafter\def\csname PY@tok@gh\endcsname{\let\PY@bf=\textbf\def\PY@tc##1{\textcolor[rgb]{0.00,0.00,0.50}{##1}}}
\expandafter\def\csname PY@tok@gu\endcsname{\let\PY@bf=\textbf\def\PY@tc##1{\textcolor[rgb]{0.50,0.00,0.50}{##1}}}
\expandafter\def\csname PY@tok@gd\endcsname{\def\PY@tc##1{\textcolor[rgb]{0.63,0.00,0.00}{##1}}}
\expandafter\def\csname PY@tok@gi\endcsname{\def\PY@tc##1{\textcolor[rgb]{0.00,0.63,0.00}{##1}}}
\expandafter\def\csname PY@tok@gr\endcsname{\def\PY@tc##1{\textcolor[rgb]{1.00,0.00,0.00}{##1}}}
\expandafter\def\csname PY@tok@ge\endcsname{\let\PY@it=\textit}
\expandafter\def\csname PY@tok@gs\endcsname{\let\PY@bf=\textbf}
\expandafter\def\csname PY@tok@gp\endcsname{\let\PY@bf=\textbf\def\PY@tc##1{\textcolor[rgb]{0.00,0.00,0.50}{##1}}}
\expandafter\def\csname PY@tok@go\endcsname{\def\PY@tc##1{\textcolor[rgb]{0.53,0.53,0.53}{##1}}}
\expandafter\def\csname PY@tok@gt\endcsname{\def\PY@tc##1{\textcolor[rgb]{0.00,0.27,0.87}{##1}}}
\expandafter\def\csname PY@tok@err\endcsname{\def\PY@bc##1{\setlength{\fboxsep}{0pt}\fcolorbox[rgb]{1.00,0.00,0.00}{1,1,1}{\strut ##1}}}
\expandafter\def\csname PY@tok@kc\endcsname{\let\PY@bf=\textbf\def\PY@tc##1{\textcolor[rgb]{0.00,0.50,0.00}{##1}}}
\expandafter\def\csname PY@tok@kd\endcsname{\let\PY@bf=\textbf\def\PY@tc##1{\textcolor[rgb]{0.00,0.50,0.00}{##1}}}
\expandafter\def\csname PY@tok@kn\endcsname{\let\PY@bf=\textbf\def\PY@tc##1{\textcolor[rgb]{0.00,0.50,0.00}{##1}}}
\expandafter\def\csname PY@tok@kr\endcsname{\let\PY@bf=\textbf\def\PY@tc##1{\textcolor[rgb]{0.00,0.50,0.00}{##1}}}
\expandafter\def\csname PY@tok@bp\endcsname{\def\PY@tc##1{\textcolor[rgb]{0.00,0.50,0.00}{##1}}}
\expandafter\def\csname PY@tok@fm\endcsname{\def\PY@tc##1{\textcolor[rgb]{0.00,0.00,1.00}{##1}}}
\expandafter\def\csname PY@tok@vc\endcsname{\def\PY@tc##1{\textcolor[rgb]{0.10,0.09,0.49}{##1}}}
\expandafter\def\csname PY@tok@vg\endcsname{\def\PY@tc##1{\textcolor[rgb]{0.10,0.09,0.49}{##1}}}
\expandafter\def\csname PY@tok@vi\endcsname{\def\PY@tc##1{\textcolor[rgb]{0.10,0.09,0.49}{##1}}}
\expandafter\def\csname PY@tok@vm\endcsname{\def\PY@tc##1{\textcolor[rgb]{0.10,0.09,0.49}{##1}}}
\expandafter\def\csname PY@tok@sa\endcsname{\def\PY@tc##1{\textcolor[rgb]{0.73,0.13,0.13}{##1}}}
\expandafter\def\csname PY@tok@sb\endcsname{\def\PY@tc##1{\textcolor[rgb]{0.73,0.13,0.13}{##1}}}
\expandafter\def\csname PY@tok@sc\endcsname{\def\PY@tc##1{\textcolor[rgb]{0.73,0.13,0.13}{##1}}}
\expandafter\def\csname PY@tok@dl\endcsname{\def\PY@tc##1{\textcolor[rgb]{0.73,0.13,0.13}{##1}}}
\expandafter\def\csname PY@tok@s2\endcsname{\def\PY@tc##1{\textcolor[rgb]{0.73,0.13,0.13}{##1}}}
\expandafter\def\csname PY@tok@sh\endcsname{\def\PY@tc##1{\textcolor[rgb]{0.73,0.13,0.13}{##1}}}
\expandafter\def\csname PY@tok@s1\endcsname{\def\PY@tc##1{\textcolor[rgb]{0.73,0.13,0.13}{##1}}}
\expandafter\def\csname PY@tok@mb\endcsname{\def\PY@tc##1{\textcolor[rgb]{0.40,0.40,0.40}{##1}}}
\expandafter\def\csname PY@tok@mf\endcsname{\def\PY@tc##1{\textcolor[rgb]{0.40,0.40,0.40}{##1}}}
\expandafter\def\csname PY@tok@mh\endcsname{\def\PY@tc##1{\textcolor[rgb]{0.40,0.40,0.40}{##1}}}
\expandafter\def\csname PY@tok@mi\endcsname{\def\PY@tc##1{\textcolor[rgb]{0.40,0.40,0.40}{##1}}}
\expandafter\def\csname PY@tok@il\endcsname{\def\PY@tc##1{\textcolor[rgb]{0.40,0.40,0.40}{##1}}}
\expandafter\def\csname PY@tok@mo\endcsname{\def\PY@tc##1{\textcolor[rgb]{0.40,0.40,0.40}{##1}}}
\expandafter\def\csname PY@tok@ch\endcsname{\let\PY@it=\textit\def\PY@tc##1{\textcolor[rgb]{0.25,0.50,0.50}{##1}}}
\expandafter\def\csname PY@tok@cm\endcsname{\let\PY@it=\textit\def\PY@tc##1{\textcolor[rgb]{0.25,0.50,0.50}{##1}}}
\expandafter\def\csname PY@tok@cpf\endcsname{\let\PY@it=\textit\def\PY@tc##1{\textcolor[rgb]{0.25,0.50,0.50}{##1}}}
\expandafter\def\csname PY@tok@c1\endcsname{\let\PY@it=\textit\def\PY@tc##1{\textcolor[rgb]{0.25,0.50,0.50}{##1}}}
\expandafter\def\csname PY@tok@cs\endcsname{\let\PY@it=\textit\def\PY@tc##1{\textcolor[rgb]{0.25,0.50,0.50}{##1}}}

\def\PYZbs{\char`\\}
\def\PYZus{\char`\_}
\def\PYZob{\char`\{}
\def\PYZcb{\char`\}}
\def\PYZca{\char`\^}
\def\PYZam{\char`\&}
\def\PYZlt{\char`\<}
\def\PYZgt{\char`\>}
\def\PYZsh{\char`\#}
\def\PYZpc{\char`\%}
\def\PYZdl{\char`\$}
\def\PYZhy{\char`\-}
\def\PYZsq{\char`\'}
\def\PYZdq{\char`\"}
\def\PYZti{\char`\~}
% for compatibility with earlier versions
\def\PYZat{@}
\def\PYZlb{[}
\def\PYZrb{]}
\makeatother


    % Exact colors from NB
    \definecolor{incolor}{rgb}{0.0, 0.0, 0.5}
    \definecolor{outcolor}{rgb}{0.545, 0.0, 0.0}



    
    % Prevent overflowing lines due to hard-to-break entities
    \sloppy 
    % Setup hyperref package
    \hypersetup{
      breaklinks=true,  % so long urls are correctly broken across lines
      colorlinks=true,
      urlcolor=urlcolor,
      linkcolor=linkcolor,
      citecolor=citecolor,
      }
    % Slightly bigger margins than the latex defaults
    
    \geometry{verbose,tmargin=1in,bmargin=1in,lmargin=1in,rmargin=1in}
    
    

    \begin{document}
    
    
    \maketitle

\paragraph{Skupina:}\label{skupina}

\begin{verbatim}
- Vojtěch Polcar (polcavoj, paralelka 108) - reprezentant
- Anna Moudrá (moudrann, paralelka 102)
\end{verbatim}

\noindent
Nejprve určíme parametry pro výpočet hodnot, které určí soubory pro
vypracování úkolů.

    \begin{Verbatim}[commandchars=\\\{\}]
{\color{incolor}In [{\color{incolor}1}]:} \PY{k+kn}{import} \PY{n+nn}{numpy} \PY{k}{as} \PY{n+nn}{np}
        \PY{k+kn}{from} \PY{n+nn}{scipy} \PY{k}{import} \PY{n}{stats}
        \PY{k+kn}{from} \PY{n+nn}{scipy}\PY{n+nn}{.}\PY{n+nn}{optimize} \PY{k}{import} \PY{n}{minimize}
        \PY{k+kn}{import} \PY{n+nn}{pandas} \PY{k}{as} \PY{n+nn}{pd}
        \PY{k+kn}{import} \PY{n+nn}{seaborn} \PY{k}{as} \PY{n+nn}{sns}
        
        \PY{k+kn}{import} \PY{n+nn}{matplotlib}\PY{n+nn}{.}\PY{n+nn}{pyplot} \PY{k}{as} \PY{n+nn}{plt}
\end{Verbatim}

    \begin{Verbatim}[commandchars=\\\{\}]
{\color{incolor}In [{\color{incolor}14}]:} \PY{n}{k} \PY{o}{=} \PY{l+m+mi}{20}
         \PY{n}{l} \PY{o}{=} \PY{l+m+mi}{6}
         \PY{n}{x} \PY{o}{=} \PY{p}{(}\PY{p}{(}\PY{n}{k}\PY{o}{*}\PY{n}{l}\PY{o}{*}\PY{l+m+mi}{23}\PY{p}{)} \PY{o}{\PYZpc{}} \PY{l+m+mi}{20}\PY{p}{)} \PY{o}{+} \PY{l+m+mi}{1}
         
         \PY{n}{str\PYZus{}X} \PY{o}{=} \PY{p}{(}\PY{l+m+mi}{3} \PY{o}{\PYZhy{}} \PY{n+nb}{len}\PY{p}{(}\PY{n+nb}{str}\PY{p}{(}\PY{n}{x}\PY{p}{)}\PY{p}{)}\PY{p}{)}\PY{o}{*}\PY{l+s+s1}{\PYZsq{}}\PY{l+s+s1}{0}\PY{l+s+s1}{\PYZsq{}}\PY{o}{+} \PY{n+nb}{str}\PY{p}{(}\PY{n}{x}\PY{p}{)} \PY{o}{+} \PY{l+s+s1}{\PYZsq{}}\PY{l+s+s1}{.txt}\PY{l+s+s1}{\PYZsq{}}
         
         \PY{n+nb}{print}\PY{p}{(}\PY{l+s+s2}{\PYZdq{}}\PY{l+s+s2}{Vstupní soubor:}\PY{l+s+s2}{\PYZdq{}}\PY{p}{,} \PY{n}{str\PYZus{}X}\PY{p}{)}
\end{Verbatim}

    \begin{Verbatim}[commandchars=\\\{\}]
Vstupní soubor: 001.txt

    \end{Verbatim}

    \begin{center}\rule{0.5\linewidth}{\linethickness}\end{center}

\section*{1) Z~datového souboru načtěte text k~analýze. Odhadněte
pravděpodobnosti písmen (včetně mezer), které se v~textu vyskytují.
Takto získané empirické rozdělení graficky
znázorněte.}

V~této úloze nejprve načteme text ze souboru k~analýze a rozdělíme ho na
jednotlivé znaky. Poté projdeme celý text a jednotlivé znaky sečteme.
Tyto hodnoty nakonec vydělíme celkovým počtem znaků, z~čehož dostaneme
pravděpodobnost výskytu znaků v~textu. Pravděpodobnost znaků je získána
ze vzorce: \[P(c)=\frac{m}{n}\] kde c je aktuální znak, m je počet
zastoupení znaku v~textu a n je celkový počet všech znaků.

    \begin{Verbatim}[commandchars=\\\{\}]
{\color{incolor}In [{\color{incolor}3}]:} \PY{n}{f} \PY{o}{=} \PY{n+nb}{open}\PY{p}{(}\PY{l+s+s1}{\PYZsq{}}\PY{l+s+s1}{./hw1\PYZhy{}source/}\PY{l+s+s1}{\PYZsq{}}\PY{o}{+}\PY{n}{str\PYZus{}X}\PY{p}{,} \PY{l+s+s1}{\PYZsq{}}\PY{l+s+s1}{r}\PY{l+s+s1}{\PYZsq{}}\PY{p}{)}
        \PY{n}{X\PYZus{}info} \PY{o}{=} \PY{n}{f}\PY{o}{.}\PY{n}{readline}\PY{p}{(}\PY{p}{)}
        \PY{n}{data\PYZus{}X} \PY{o}{=} \PY{n}{f}\PY{o}{.}\PY{n}{read}\PY{p}{(}\PY{p}{)}
        \PY{n}{f}\PY{o}{.}\PY{n}{close}\PY{p}{(}\PY{p}{)}
        
        \PY{n+nb}{print}\PY{p}{(}\PY{l+s+s1}{\PYZsq{}}\PY{l+s+s1}{Název textu:}\PY{l+s+s1}{\PYZsq{}}\PY{p}{,} \PY{n}{X\PYZus{}info}\PY{p}{)}
\end{Verbatim}

    \begin{Verbatim}[commandchars=\\\{\}]
Název textu: Pierrot, Dog of Belgium, by Walter A. Dyer


    \end{Verbatim}

    \begin{Verbatim}[commandchars=\\\{\}]
{\color{incolor}In [{\color{incolor}4}]:} \PY{n}{x\PYZus{}uni} \PY{o}{=} \PY{n}{np}\PY{o}{.}\PY{n}{array}\PY{p}{(}\PY{n+nb}{list}\PY{p}{(}\PY{n}{data\PYZus{}X}\PY{p}{[}\PY{p}{:}\PY{p}{]}\PY{p}{)}\PY{p}{)}
        \PY{n}{x\PYZus{}uni} \PY{o}{=} \PY{n}{np}\PY{o}{.}\PY{n}{unique}\PY{p}{(}\PY{n}{x\PYZus{}uni}\PY{p}{,} \PY{n}{return\PYZus{}counts}\PY{o}{=}\PY{k+kc}{True}\PY{p}{)}
        \PY{n+nb}{print}\PY{p}{(}\PY{n}{x\PYZus{}uni}\PY{p}{)}
\end{Verbatim}

    \begin{Verbatim}[commandchars=\\\{\}]
(array([' ', 'a', 'b', 'c', 'd', 'e', 'f', 'g', 'h', 'i', 'j', 'k', 'l',
       'm', 'n', 'o', 'p', 'q', 'r', 's', 't', 'u', 'v', 'w', 'x', 'y', 'z'],
        dtype='<U1'), 
 array([1199, 438, 96, 102, 303, 659, 137, 141, 337, 297, 15, 52, 249, 
       109, 355, 399, 91, 6,  350,  291,  436,  159, 30,  117, 1, 83, 6]))

    \end{Verbatim}

    \begin{Verbatim}[commandchars=\\\{\}]
{\color{incolor}In [{\color{incolor}5}]:} \PY{n}{chars\PYZus{}x} \PY{o}{=} \PY{n}{pd}\PY{o}{.}\PY{n}{DataFrame}\PY{p}{(}\PY{p}{)}
        \PY{n}{chars\PYZus{}x}\PY{p}{[}\PY{l+s+s1}{\PYZsq{}}\PY{l+s+s1}{char}\PY{l+s+s1}{\PYZsq{}}\PY{p}{]} \PY{o}{=} \PY{n}{x\PYZus{}uni}\PY{p}{[}\PY{l+m+mi}{0}\PY{p}{]}
        \PY{n}{chars\PYZus{}x}\PY{p}{[}\PY{l+s+s1}{\PYZsq{}}\PY{l+s+s1}{count}\PY{l+s+s1}{\PYZsq{}}\PY{p}{]} \PY{o}{=} \PY{n}{x\PYZus{}uni}\PY{p}{[}\PY{l+m+mi}{1}\PY{p}{]}
        \PY{n}{chars\PYZus{}x}\PY{p}{[}\PY{l+s+s1}{\PYZsq{}}\PY{l+s+s1}{prob}\PY{l+s+s1}{\PYZsq{}}\PY{p}{]} \PY{o}{=} \PY{n}{chars\PYZus{}x}\PY{p}{[}\PY{l+s+s1}{\PYZsq{}}\PY{l+s+s1}{count}\PY{l+s+s1}{\PYZsq{}}\PY{p}{]}\PY{o}{/}\PY{p}{(}\PY{n}{np}\PY{o}{.}\PY{n}{sum}\PY{p}{(}\PY{n}{x\PYZus{}uni}\PY{p}{[}\PY{l+m+mi}{1}\PY{p}{]}\PY{p}{)}\PY{p}{)}
        
        \PY{n}{chars\PYZus{}x}\PY{o}{.}\PY{n}{char} \PY{o}{=} \PY{n}{chars\PYZus{}x}\PY{o}{.}\PY{n}{char}\PY{o}{.}\PY{n}{replace}\PY{p}{(}\PY{l+s+s2}{\PYZdq{}}\PY{l+s+s2}{ }\PY{l+s+s2}{\PYZdq{}}\PY{p}{,} \PY{l+s+s1}{\PYZsq{}}\PY{l+s+s1}{space}\PY{l+s+s1}{\PYZsq{}}\PY{p}{)}
\end{Verbatim}

    \begin{Verbatim}[commandchars=\\\{\}]
{\color{incolor}In [{\color{incolor}6}]:} \PY{n}{colors} \PY{o}{=} \PY{p}{[}\PY{l+s+s2}{\PYZdq{}}\PY{l+s+s2}{\PYZsh{}3778bf}\PY{l+s+s2}{\PYZdq{}}\PY{p}{,}\PY{l+s+s2}{\PYZdq{}}\PY{l+s+s2}{\PYZsh{}feb308}\PY{l+s+s2}{\PYZdq{}}\PY{p}{]}
        \PY{n}{plt}\PY{o}{.}\PY{n}{figure}\PY{p}{(}\PY{n}{figsize}\PY{o}{=}\PY{p}{(}\PY{l+m+mi}{14}\PY{p}{,}\PY{l+m+mi}{5}\PY{p}{)}\PY{p}{)}
        \PY{n}{sns}\PY{o}{.}\PY{n}{barplot}\PY{p}{(}\PY{n}{x}\PY{o}{=}\PY{l+s+s1}{\PYZsq{}}\PY{l+s+s1}{char}\PY{l+s+s1}{\PYZsq{}}\PY{p}{,} \PY{n}{y}\PY{o}{=}\PY{l+s+s1}{\PYZsq{}}\PY{l+s+s1}{prob}\PY{l+s+s1}{\PYZsq{}}\PY{p}{,} \PY{n}{data}\PY{o}{=}\PY{n}{chars\PYZus{}x}\PY{p}{,} \PY{n}{color}\PY{o}{=}\PY{n}{colors}\PY{p}{[}\PY{l+m+mi}{0}\PY{p}{]}\PY{p}{,} \PY{n}{saturation}\PY{o}{=}\PY{l+m+mi}{1}\PY{p}{)}
        \PY{n}{plt}\PY{o}{.}\PY{n}{title}\PY{p}{(}\PY{l+s+s2}{\PYZdq{}}\PY{l+s+s2}{Rozdělení znaků v textu}\PY{l+s+s2}{\PYZdq{}}\PY{p}{)}\PY{p}{,}\PY{n}{plt}\PY{o}{.}\PY{n}{xlabel}\PY{p}{(}\PY{l+s+s2}{\PYZdq{}}\PY{l+s+s2}{znak}\PY{l+s+s2}{\PYZdq{}}\PY{p}{)}\PY{p}{,}
        \PY{n}{plt}\PY{o}{.}\PY{n}{ylabel}\PY{p}{(}\PY{l+s+s2}{\PYZdq{}}\PY{l+s+s2}{pravděpodobnost výskytu}\PY{l+s+s2}{\PYZdq{}}\PY{p}{)}
        \PY{n}{plt}\PY{o}{.}\PY{n}{show}\PY{p}{(}\PY{p}{)}
\end{Verbatim}

    \begin{center}
    \adjustimage{max size={0.9\linewidth}{0.9\paperheight}}{output_7_0.png}
    \end{center}
    { \hspace*{\fill} \\}
    
\subsection*{Diskuze k~výsledku}\label{diskuze-k-vuxfdsledku}

Z~grafu můžeme vidět, že rozdělení písmen odpovídá tomu, že je text
v~anglickém jazyce. Pokud nebudeme počítat mezery, tak jsou v~něm
nejčastějšími znaky právě písmena e, t, a, což potvrzuje i graf.

\paragraph{Pro další body předpokládejme, že je text vygenerován
z~homogenního markovského řetězce s~diskrétním
časem.}\label{pro-dalux161uxed-body-pux159edpokluxe1dejme-ux17ee-je-text-vygenerovuxe1n-z-homogennuxedho-markovskuxe9ho-ux159etux11bzce-s-diskruxe9tnuxedm-ux10dasem.}

\section*{2) Za tohoto předpokladu odhadněte matici
přechodu.}\label{za-tohoto-pux159edpokladu-odhadnux11bte-matici-pux159echodu.}

Ze zadaného textu odhadneme matici přechodu. Protože předpokládáme, že je
text vygenerován z~homogenního markovského řetězce, získáme matici tak,
že projdeme text a pro každý znak se podíváme na jeho následníka. To
můžeme udělat protože předchozí předpoklad nám říká, že nezávisí na tom,
jaké následníky měl znak dříve, ale zajímá nás pouze aktuální stav.

V~řešení níže projdeme zadaný text a pro každý znak vytvoříme
pravděpodobnost přechodu na všechny znaky. Pravděpodobnost každého
přechodu odhadneme pomocí vzorce:
\[P(M_{i+1}= d | M_i = c)=\frac{m}{n}\] kde c je aktuální znak, d je
následující znak, m je počet výskytů znaku d za znakem c a n je počet
všech výskytů znaku c (neboli počet všech následníků).

    \begin{Verbatim}[commandchars=\\\{\}]
{\color{incolor}In [{\color{incolor}17}]:} \PY{n}{chars} \PY{o}{=} \PY{n}{chars\PYZus{}x}\PY{o}{.}\PY{n}{char}\PY{o}{.}\PY{n}{unique}\PY{p}{(}\PY{p}{)}
         \PY{c+c1}{\PYZsh{}print(chars)}
         \PY{n}{P} \PY{o}{=} \PY{n}{pd}\PY{o}{.}\PY{n}{DataFrame}\PY{p}{(}\PY{n}{index}\PY{o}{=}\PY{n}{chars}\PY{p}{,} \PY{n}{columns}\PY{o}{=}\PY{n}{chars}\PY{p}{)}
         
         
         \PY{k}{def} \PY{n+nf}{fillP}\PY{p}{(}\PY{n}{df}\PY{p}{,} \PY{n}{text}\PY{p}{,} \PY{n}{alphabet}\PY{p}{)}\PY{p}{:}
           \PY{k}{for} \PY{n}{a} \PY{o+ow}{in} \PY{n}{alphabet}\PY{p}{:}
              \PY{n}{char} \PY{o}{=} \PY{n}{a}
              \PY{k}{if} \PY{n}{char} \PY{o}{==} \PY{l+s+s1}{\PYZsq{}}\PY{l+s+s1}{space}\PY{l+s+s1}{\PYZsq{}}\PY{p}{:}
                \PY{n}{char} \PY{o}{=} \PY{l+s+s2}{\PYZdq{}}\PY{l+s+s2}{ }\PY{l+s+s2}{\PYZdq{}}
              \PY{n}{indexes} \PY{o}{=} \PY{p}{[}\PY{n+nb}{int}\PY{p}{(}\PY{n}{index}\PY{o}{+}\PY{l+m+mi}{1}\PY{p}{)} \PY{k}{for} \PY{n}{index}\PY{p}{,} \PY{n}{value} \PY{o+ow}{in} \PY{n+nb}{enumerate}\PY{p}{(}\PY{n}{text}\PY{p}{)} \PY{k}{if} \PY{n}{value} \PY{o}{==} \PY{n}{char}\PY{p}{]}
              \PY{n}{found} \PY{o}{=} \PY{p}{[}\PY{n}{text}\PY{p}{[}\PY{n}{i}\PY{p}{]} \PY{k}{for} \PY{n}{i} \PY{o+ow}{in} \PY{n}{indexes} \PY{k}{if} \PY{n}{i} \PY{o}{\PYZlt{}} \PY{n+nb}{len}\PY{p}{(}\PY{n}{text}\PY{p}{)}\PY{p}{]}
              \PY{k}{for} \PY{n}{b} \PY{o+ow}{in} \PY{n}{df}\PY{o}{.}\PY{n}{columns}\PY{p}{:}
                 \PY{n}{after\PYZus{}char} \PY{o}{=} \PY{n}{b}
                 \PY{k}{if} \PY{n}{after\PYZus{}char} \PY{o}{==} \PY{l+s+s1}{\PYZsq{}}\PY{l+s+s1}{space}\PY{l+s+s1}{\PYZsq{}}\PY{p}{:}
                     \PY{n}{after\PYZus{}char} \PY{o}{=} \PY{l+s+s2}{\PYZdq{}}\PY{l+s+s2}{ }\PY{l+s+s2}{\PYZdq{}}
                 \PY{k}{if}\PY{p}{(}\PY{n}{after\PYZus{}char} \PY{o+ow}{in} \PY{n+nb}{set}\PY{p}{(}\PY{n}{found}\PY{p}{)}\PY{p}{)}\PY{p}{:}
                     \PY{n}{after\PYZus{}char\PYZus{}prob} \PY{o}{=} \PY{n}{found}\PY{o}{.}\PY{n}{count}\PY{p}{(}\PY{n}{after\PYZus{}char}\PY{p}{)}\PY{o}{/}\PY{n+nb}{len}\PY{p}{(}\PY{n}{found}\PY{p}{)}
                     \PY{n}{df}\PY{p}{[}\PY{n}{b}\PY{p}{]}\PY{p}{[}\PY{n}{a}\PY{p}{]} \PY{o}{=} \PY{n}{after\PYZus{}char\PYZus{}prob}
                 \PY{k}{else}\PY{p}{:}
                     \PY{n}{df}\PY{p}{[}\PY{n}{b}\PY{p}{]}\PY{p}{[}\PY{n}{a}\PY{p}{]} \PY{o}{=} \PY{l+m+mi}{0}
                     
                 
                     
         \PY{n}{fillP}\PY{p}{(}\PY{n}{P}\PY{p}{,} \PY{n}{data\PYZus{}X}\PY{p}{,}\PY{n}{chars}\PY{p}{)}
         \PY{c+c1}{\PYZsh{}display(P.head())}
         \PY{n+nb}{print}\PY{p}{(}\PY{l+s+s2}{\PYZdq{}}\PY{l+s+s2}{Test (součet řádků): }\PY{l+s+s2}{\PYZdq{}}\PY{p}{)}
         \PY{n+nb}{print}\PY{p}{(}\PY{n}{P}\PY{o}{.}\PY{n}{sum}\PY{p}{(}\PY{n}{axis}\PY{o}{=}\PY{l+m+mi}{1}\PY{p}{)}\PY{p}{)}
\end{Verbatim}

    \begin{Verbatim}[commandchars=\\\{\}]
Test (součet řádků): 
space    1.0
a        1.0
b        1.0
c        1.0
d        1.0
e        1.0
f        1.0
g        1.0
h        1.0
i        1.0
j        1.0
k        1.0
l        1.0
m        1.0
n        1.0
o        1.0
p        1.0
q        1.0
r        1.0
s        1.0
t        1.0
u        1.0
v        1.0
w        1.0
x        1.0
y        1.0
z        1.0
dtype: float64

    \end{Verbatim}

    \begin{Verbatim}[commandchars=\\\{\}]
{\color{incolor}In [{\color{incolor}8}]:} \PY{n}{P}\PY{o}{.}\PY{n}{to\PYZus{}csv}\PY{p}{(}\PY{l+s+s2}{\PYZdq{}}\PY{l+s+s2}{matice\PYZus{}prechodu.csv}\PY{l+s+s2}{\PYZdq{}}\PY{p}{)}
\end{Verbatim}

\subsection*{Diskuze k~výsledku}\label{diskuze-k-vuxfdsledku}

Výsledkem předchozích výpočtů je matice o~velikosti 27x27. Pro
přehlednost je matice vyexportovaná do souboru ``matice\_prechodu.csv''.
Pro kontrolu byl proveden i součet pravděpodobností jednotlivých řádků.
Ten správně vyšel pro každý řádek roven 1.

\section*{3) Na základě matice z~předchozího bodu najděte stacionární
rozdělení tohoto
řetězce.}\label{na-zuxe1kladux11b-matice-z-pux159edchozuxedho-bodu-najdux11bte-stacionuxe1rnuxed-rozdux11blenuxed-tohoto-ux159etux11bzce.}

Stacionární rozdělení řetězce najdeme na základě matice z~předchozího
příkladu. Stacionární rozdělení \(\pi\) je řešením soustavy rovnic
\( \boldsymbol{\pi} = \boldsymbol{\pi}*\mathbf{P} \) a \(\sum_{i=0}^{26} \pi_i = 1 \).. Tuto soustavu rovnic si upravíme na: \[ \boldsymbol{\pi}*(\mathbf{I} - \mathbf{P}) = 0 \], kde \(\mathbf{I} \)  je jednotková matice.

Soustavu nyní můžeme napsat ve tvaru \( \mathbf{A}\boldsymbol{x} = \boldsymbol{b} \), kde \( \boldsymbol{b}\) je \((0,\dots,0,1)\) a \(\mathbf{A}\)~je matice \( \boldsymbol{\pi}*(\mathbf{I} - \mathbf{P}) \) s~přidaným
posledním řádkem, reprezentujícím
\( \boldsymbol{\pi} - (\pi_0, \pi_1,\dots,\pi_{26}) \). 
Tuto soustavu rovnic vyřešíme a výsledkem bude  stacionárního rozdělení \( \boldsymbol{\pi} \).

    \begin{Verbatim}[commandchars=\\\{\}]
{\color{incolor}In [{\color{incolor}13}]:} \PY{l+s+sd}{\PYZdq{}\PYZdq{}\PYZdq{} }
         \PY{l+s+sd}{pi = pi*P }
         \PY{l+s+sd}{soustava rovnic kde vektor pi je resenim soustavy:}
         \PY{l+s+sd}{        pi( I\PYZhy{} P ) == 0   upravena rovnice}
         \PY{l+s+sd}{        sum(pi) == 1}
         \PY{l+s+sd}{        }
         \PY{l+s+sd}{        tuto soustavu muzeme napsat ve tvaru Ax = b}
         \PY{l+s+sd}{        kde b.T = [0,..,0,1]}
         \PY{l+s+sd}{        a A= [[}
         \PY{l+s+sd}{                }
         \PY{l+s+sd}{                    pi( I\PYZhy{} P )}
         \PY{l+s+sd}{                        }
         \PY{l+s+sd}{                                  ]}
         \PY{l+s+sd}{                [ pi0 + .. + pi26 ]]}
         \PY{l+s+sd}{\PYZdq{}\PYZdq{}\PYZdq{}}
         \PY{k}{def} \PY{n+nf}{stationary}\PY{p}{(}\PY{n}{P}\PY{p}{)}\PY{p}{:}
             \PY{n}{rows} \PY{o}{=} \PY{n}{P}\PY{o}{.}\PY{n}{shape}\PY{p}{[}\PY{l+m+mi}{0}\PY{p}{]} \PY{c+c1}{\PYZsh{} velikost vektoru pi}
             \PY{n}{a} \PY{o}{=} \PY{n}{np}\PY{o}{.}\PY{n}{eye}\PY{p}{(}\PY{n}{rows}\PY{p}{)} \PY{o}{\PYZhy{}} \PY{n}{P} \PY{c+c1}{\PYZsh{} I\PYZhy{} P \PYZsh{} vektor levych stran}
             \PY{n}{a} \PY{o}{=} \PY{n}{np}\PY{o}{.}\PY{n}{vstack}\PY{p}{(} \PY{p}{(}\PY{n}{a}\PY{o}{.}\PY{n}{T}\PY{p}{,} \PY{n}{np}\PY{o}{.}\PY{n}{ones}\PY{p}{(}\PY{n}{rows}\PY{p}{)}\PY{p}{)} \PY{p}{)} \PY{c+c1}{\PYZsh{} pridani sum(pi) == 1}
             \PY{c+c1}{\PYZsh{} v promenne a je ted matice A v soustave Ax = b}
             \PY{n}{b\PYZus{}vect} \PY{o}{=} \PY{n}{np}\PY{o}{.}\PY{n}{zeros}\PY{p}{(}\PY{n}{rows}\PY{o}{+}\PY{l+m+mi}{1}\PY{p}{)} \PY{c+c1}{\PYZsh{} vektor pravych stran b}
             \PY{n}{b\PYZus{}vect}\PY{p}{[}\PY{o}{\PYZhy{}}\PY{l+m+mi}{1}\PY{p}{]} \PY{o}{=} \PY{l+m+mi}{1} \PY{c+c1}{\PYZsh{} 0 pro kazdou rovnici  pi( I \PYZhy{} P ) a 1 pro sum(pi) == 1}
             \PY{n}{pi} \PY{o}{=} \PY{n}{np}\PY{o}{.}\PY{n}{linalg}\PY{o}{.}\PY{n}{lstsq}\PY{p}{(} \PY{n}{a}\PY{o}{.}\PY{n}{astype}\PY{p}{(}\PY{n+nb}{float}\PY{p}{)}\PY{p}{,} \PY{n}{np}¨\PY{o}{.}\PY{n}{matrix}\PY{p}{(}\PY{n}{b\PYZus{}vect}\PY{p}{,} \PY{n}{dtype} \PY{o}{=} \PY{l+s+s1}{\PYZsq{}}\PY{l+s+s1}{float}\PY{l+s+s1}{\PYZsq{}}\PY{p}{)}\PY{o}{.}\PY{n}{T}
             \PY{p}{,} \PY{n}{rcond}\PY{o}{=}\PY{k+kc}{None} \PY{p}{)} \PY{c+c1}{\PYZsh{}reseni sustavy rovnic}
             \PY{k}{return} \PY{p}{[}\PY{n}{i}\PY{p}{[}\PY{l+m+mi}{0}\PY{p}{]} \PY{k}{for} \PY{n}{i} \PY{o+ow}{in} \PY{n}{pi}\PY{p}{[}\PY{l+m+mi}{0}\PY{p}{]}\PY{o}{.}\PY{n}{tolist}\PY{p}{(}\PY{p}{)}\PY{p}{]} \PY{c+c1}{\PYZsh{}prevod 2D listu do vektoru reseni}
             
         
         \PY{n}{P\PYZus{}matrix} \PY{o}{=} \PY{n}{P}\PY{o}{.}\PY{n}{values}
         \PY{n}{p\PYZus{}stationary} \PY{o}{=} \PY{n}{stationary}\PY{p}{(}\PY{n}{P}\PY{p}{)}
         \PY{n+nb}{print}\PY{p}{(}\PY{l+s+s2}{\PYZdq{}}\PY{l+s+s2}{Stationary:}\PY{l+s+s2}{\PYZdq{}}\PY{p}{,}\PY{n}{p\PYZus{}stationary}\PY{p}{)}
         
         \PY{n+nb}{print}\PY{p}{(}\PY{l+s+s2}{\PYZdq{}}\PY{l+s+se}{\PYZbs{}n}\PY{l+s+s2}{Kontrolní součet:}\PY{l+s+s2}{\PYZdq{}}\PY{p}{,}\PY{n}{np}\PY{o}{.}\PY{n}{sum}\PY{p}{(}\PY{n}{p\PYZus{}stationary}\PY{p}{)}\PY{p}{)} \PY{c+c1}{\PYZsh{}kontrolni soucet}
\end{Verbatim}

    \begin{Verbatim}[commandchars=\\\{\}]
Stationary: [0.18570145528394705, 0.06783413051436557, 0.014868368795939646,
 	0.015796940837522255, 0.04692576751394258, 0.10205425965931908, 
 	0.021216882050768715, 0.021843746900416038, 0.05215764575441392, 
 	0.04599409450396163, 0.00232320419454483, 0.008053552592696462, 
 	0.038557951783603245, 0.01688022014058139, 0.054979412400881736, 
 	0.061783587274558446, 0.014093962454750886, 0.0009294701178297338, 
 	0.054219534943122406, 0.04508217464800904, 0.06736804148266043, 
 	0.02462805399717019, 0.004645995127833057, 0.01812015989003305, 
 	0.00015486230600811335, 0.012857372099992197, 0.0009291527311291806]

Kontrolní součet: 1.0000000000000009

    \end{Verbatim}

    \begin{Verbatim}[commandchars=\\\{\}]
{\color{incolor}In [{\color{incolor}18}]:} \PY{c+c1}{\PYZsh{}print(chars)}
         \PY{n}{count} \PY{o}{=} \PY{n+nb}{len}\PY{p}{(}\PY{n}{chars}\PY{p}{)}
         \PY{n}{df} \PY{o}{=} \PY{n}{pd}\PY{o}{.}\PY{n}{DataFrame}\PY{p}{(}\PY{n}{data} \PY{o}{=} \PY{p}{\PYZob{}}\PY{l+s+s1}{\PYZsq{}}\PY{l+s+s1}{char}\PY{l+s+s1}{\PYZsq{}}\PY{p}{:} \PY{n+nb}{list}\PY{p}{(}\PY{n}{chars}\PY{p}{)}\PY{o}{+}\PY{n+nb}{list}\PY{p}{(}\PY{n}{chars}\PY{p}{)}\PY{p}{,}\PY{l+s+s1}{\PYZsq{}}\PY{l+s+s1}{values}\PY{l+s+s1}{\PYZsq{}}\PY{p}{:}
         \PY{n+nb}{list}\PY{p}{(}\PY{n}{chars\PYZus{}x}\PY{o}{.}\PY{n}{prob}\PY{p}{)}\PY{o}{+}\PY{n}{p\PYZus{}stationary}\PY{p}{,} \PY{l+s+s1}{\PYZsq{}}\PY{l+s+s1}{kind}\PY{l+s+s1}{\PYZsq{}}\PY{p}{:} \PY{n+nb}{list}\PY{p}{(}\PY{p}{[}\PY{l+s+s1}{\PYZsq{}}\PY{l+s+s1}{data}\PY{l+s+s1}{\PYZsq{}}\PY{p}{]}\PY{o}{*}\PY{n}{count}\PY{o}{+}\PY{p}{[}\PY{l+s+s1}{\PYZsq{}}\PY{l+s+s1}{stationary}\PY{l+s+s1}{\PYZsq{}}\PY{p}{]}\PY{o}{*}\PY{n}{count}\PY{p}{)}\PY{p}{\PYZcb{}}\PY{p}{)}
         \PY{n}{plt}\PY{o}{.}\PY{n}{figure}\PY{p}{(}\PY{n}{figsize}\PY{o}{=}\PY{p}{(}\PY{l+m+mi}{14}\PY{p}{,} \PY{l+m+mi}{5}\PY{p}{)}\PY{p}{)}
         \PY{n}{sns}\PY{o}{.}\PY{n}{barplot}\PY{p}{(}\PY{n}{x}\PY{o}{=}\PY{l+s+s1}{\PYZsq{}}\PY{l+s+s1}{char}\PY{l+s+s1}{\PYZsq{}}\PY{p}{,} \PY{n}{hue}\PY{o}{=}\PY{l+s+s2}{\PYZdq{}}\PY{l+s+s2}{kind}\PY{l+s+s2}{\PYZdq{}}\PY{p}{,} \PY{n}{y}\PY{o}{=}\PY{l+s+s2}{\PYZdq{}}\PY{l+s+s2}{values}\PY{l+s+s2}{\PYZdq{}}\PY{p}{,} 
         \PY{n}{data}\PY{o}{=}\PY{n}{df}\PY{p}{,} \PY{n}{palette} \PY{o}{=} \PY{n}{colors}\PY{p}{,} \PY{n}{saturation}\PY{o}{=}\PY{l+m+mf}{1.}\PY{p}{)}
         \PY{n}{plt}\PY{o}{.}\PY{n}{title}\PY{p}{(}\PY{l+s+s2}{\PYZdq{}}\PY{l+s+s2}{Rozdělení znaků v textu a stacionární rozdělení}\PY{l+s+s2}{\PYZdq{}}\PY{p}{)}\PY{p}{,}
         \PY{n}{plt}\PY{o}{.}\PY{n}{xlabel}\PY{p}{(}\PY{l+s+s2}{\PYZdq{}}\PY{l+s+s2}{znak}\PY{l+s+s2}{\PYZdq{}}\PY{p}{)}\PY{p}{,}\PY{n}{plt}\PY{o}{.}\PY{n}{ylabel}\PY{p}{(}\PY{l+s+s2}{\PYZdq{}}\PY{l+s+s2}{pravděpodobnost výskytu}\PY{l+s+s2}{\PYZdq{}}\PY{p}{)}
         \PY{n}{plt}\PY{o}{.}\PY{n}{show}\PY{p}{(}\PY{p}{)}
\end{Verbatim}

    \begin{center}
    \adjustimage{max size={0.9\linewidth}{0.9\paperheight}}{output_15_0.png}
    \end{center}
    { \hspace*{\fill} \\}
    
\subsection*{Diskuze k~výsledku}\label{diskuze-k-vuxfdsledku}

Pro výsledný vektor stacionárního rozdělení jsme provedli kontrolní
součet, zda se \(\sum_{i=0}^{26} \pi_i\) rovná 1. Malá odchylka
1.0000000000000009 je způsobena sčítáním malých čísel s~omezenou
desetinnou přesností. Výsledný \(\boldsymbol{\pi}\) je skutečně stacionárním
rozdělením.


\section*{4) Porovnejte stacionární rozdělení se získaným
empirickým rozdělením. Tj. na hladině 5\% otestujte hypotézu, že se
empirické rozdělení z~bodu 1 rovná stacionárnímu
rozdělení.}\label{porovnejte-stacionuxe1rnuxed-rozdux11blenuxed-se-zuxedskanuxfdm-empirickuxfdm-rozdux11blenuxedm.-tj.-na-hladinux11b-5-otestujte-hypotuxe9zu-ux17ee-se-empirickuxe9-rozdux11blenuxed-z-bodu-1-rovnuxe1-stacionuxe1rnuxedmu-rozdux11blenuxed.}
Pro otestování hypotézy použijeme test dobré shody \(\chi^2\) při
známých parametrech:
\[\chi^2 = \sum_{i=1}^k \frac{(N_i - np_i)^2}{np_i}\]

\noindent
s~kritickým oborem

\[\chi^2 \geq \chi^2_{\alpha,k-1}\]

\noindent
Na hladině významnosti \(5\%\) budeme testovat hypotézu: 

- Rozdělení se rovnají: \(H_0\): \(p'=p\)

- Rozdělení jsou rozdílná: \(H_A\): \(p'\neq p\)

    \begin{Verbatim}[commandchars=\\\{\}]
{\color{incolor}In [{\color{incolor}19}]:} \PY{n}{s}\PY{p}{,}\PY{n}{p} \PY{o}{=} \PY{n}{stats}\PY{o}{.}\PY{n}{chisquare}\PY{p}{(}\PY{n}{chars\PYZus{}x}\PY{o}{.}\PY{n}{prob}\PY{p}{,} \PY{n}{p\PYZus{}stationary}\PY{p}{)}
         \PY{n+nb}{print}\PY{p}{(}\PY{l+s+s2}{\PYZdq{}}\PY{l+s+s2}{statistika:}\PY{l+s+s2}{\PYZdq{}}\PY{p}{,}\PY{l+s+s2}{\PYZdq{}}\PY{l+s+si}{\PYZpc{}.8f}\PY{l+s+s2}{\PYZdq{}} \PY{o}{\PYZpc{}} \PY{n}{s}\PY{p}{)}
         \PY{n+nb}{print}\PY{p}{(}\PY{l+s+s2}{\PYZdq{}}\PY{l+s+s2}{p\PYZus{}value:}\PY{l+s+s2}{\PYZdq{}}\PY{p}{,}\PY{n}{p}\PY{p}{)}
\end{Verbatim}

    \begin{Verbatim}[commandchars=\\\{\}]
statistika: 0.00000037
p\_value: 1.0

    \end{Verbatim}

\subsection*{Diskuze k~výsledku}\label{diskuze-k-vuxfdsledku}

Pokud je hladina významnosti \(\alpha\) větší než \(p\) hodnota testu,
zamítáme \(H_0\) ve prospěch \(H_A\) na hladině \(\alpha\). V~našem
případě je \(p\) hodnota testu rovna 1, tedy hypotézu \(H_0\)
\textbf{nezamítáme} a nezamítli bychom ji ani na hladině významnosti
blížící se \(100\%\). Stejně tak můžeme porovnat Pearsonovu statistiku
s~hodnotou z~tabulek kde \(\chi^2_{0.05,26} = 38.885\) a opět vidíme, že
se k~rozhodnutí zamítnutí ani neblížíme.


    % Add a bibliography block to the postdoc
    
    
    
    \end{document}
